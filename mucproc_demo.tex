\documentclass[12pt]{article}
\usepackage[utf8]{inputenc}
\usepackage{amsmath}
\usepackage{amsfonts}
\usepackage{amssymb}
\usepackage{hyperref}

\title{N-gram Language Models}
\author{
  Felix \\
  \and
  Borisov Timofei \\
  \and
  Simon \\
}
\date{\today}

\begin{document}

% Title page
\maketitle

% Abstract
\begin{abstract}
This is a brief summary of your essay. It should be concise and informative.
\end{abstract}

% Table of contents
\tableofcontents
\newpage

% Sections
\section{Introduction}
This is the introduction section where you provide background information on your topic and outline the structure of your essay.

\section{Main Section 1 (FELIX)}
\subsection{Subsection 1.1}
Here you can start discussing the details of your first main point. For example, algorithms are a fundamental part of computer science and understanding them is crucial for any software developer. As stated in \cite{cormen2009}, algorithms are essential for efficient problem solving in computing.

\subsection{Subsection 1.2}
Continue with further details and analysis related to your first main point.

\section{Fortgeschrittene Konzepte und Techniken in N-Gramm-Modellen (Borisov Timofei)}
\subsection{Was ist un-seen N-Grams?}
Discuss the details of your second main point.

\subsection{Smoothing Techniques.}
Further details and analysis related to your second main point.

\subsection{Vergleich von N-Grammen und neuronalen Netzen.}
Further details and analysis related to your second main point.

\section{Main Section 3 (SIMON)}
\subsection{Subsection 3.1}
Discuss the details of your third main point.
\subsection{Subsection 3.2}
Further details and analysis related to your third main point.

\section{Conclusion}
Summarize the main points discussed in your essay and provide your final thoughts.

% Bibliography
\newpage
\begin{thebibliography}{9}

\bibitem{cormen2009}
Thomas H. Cormen, Charles E. Leiserson, Ronald L. Rivest, and Clifford Stein, \textit{Introduction to Algorithms}, 3rd Edition, MIT Press, 2009.

\bibitem{example2}
Author Name, ``Article Title,'' \textit{Journal Name}, Volume(Issue), pages, Year.

\bibitem{example3}
Author Name, ``Webpage Title,'' \url{http://example.com}, Accessed: Date.

\end{thebibliography}

\end{document}
